\documentclass{exam}

\usepackage{ifxetex}
\ifxetex
    \usepackage{unicode-math}
    \usepackage{fontspec}
\else
    \usepackage{lmodern}
    \usepackage[utf8]{inputenc}
    \usepackage[T1]{fontenc}
    \DeclareUnicodeCharacter{3BB}{\lambda}
\fi

\usepackage[letterpaper,margin=1in]{geometry}

\title{Lambda Calculus Homework}
\author{}

\makeatletter
\renewcommand\@maketitle{%
\begin{center}
    \bfseries
    \centering
    \LARGE CSCI-400: \@title
\end{center}
\bigskip
\begin{center}
\makebox[0.8\textwidth]{Name:\enspace\hrulefill\hrulefill\enspace{}CWID:\enspace\hrulefill}
\end{center}
\bigskip
\begin{itemize}
    \item A \textbf{printed} copy is due in class on Thursday, February 21.
    \item Handwritten or typed answers are acceptable. If handwritten,
        please write clear and legibly.
\end{itemize}
}
\makeatother
\cfoot{Page \thepage\ of \numpages}

\begin{document}
\maketitle
\thispagestyle{foot}

\begin{questions}
    \question[50]
        Fully beta reduce each of the following lambda terms. Check your
        answer using the \texttt{lc} tool shown in class. \textbf{Do not use
        any shorthands in your evaluation}.

        \textbf{Note:} it is acceptable to skip currying steps. You should not
        skip any other steps: show all of you work!

        \begin{parts}
            \part $(λp.λq.pqp)(λx.λy.y)(λx.λy.x)$

            \vspace{\stretch{2}}

            \part $(λp.λq.ppq)(λx.λy.y)(λx.λy.x)$

            \vspace{\stretch{2}}

            \part $(λp.p(λx.λy.y)(λx.λy.x))(λx.λy.y)$

            \vspace{\stretch{2}}

            \clearpage

            \part $(λp.λa.λb.pab)(λx.λy.y)(λf.(λx.fx))(λf.(λx.f(fx)))$

            \vspace{\stretch{2}}

            \part $(λc.c(λx.λy.y))(λx.λy.λf.fxy)(λf.λx.f(fx))(λf.λx.f(f(fx)))$

            \vspace{\stretch{2}}
        \end{parts}

    \question[20]
        Identify which variables are free and which are bound in each of the
        lambda terms:

        \begin{parts}
            \part $λx.xy$

            \vspace{\stretch{1}}

            \part $(λx.x)m$

            \vspace{\stretch{1}}
        \end{parts}

    \newpage
    \bonusquestion[10]
        Write a lambda calculus abstraction which, when a Church numeral $n$ is
        applied, evaluates to the Church numeral $n - 1$. Note that this is the
        inverse of the \texttt{SUCC} function. Then, explain how your
        abstraction works and provide an example for the Church numeral 5. (Note
        that this abstraction need not handle 0.)


\end{questions}

\end{document}
