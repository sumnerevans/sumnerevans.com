\documentclass{exam}
\usepackage[letterpaper,margin=1in]{geometry}
\usepackage{minted}
\setminted{autogobble,python3,mathescape,linenos,frame=lines,framesep=2mm,fontsize=\footnotesize}
\setlength{\parindent}{0cm}

\title{Homework 2}
\author{}

\makeatletter
\renewcommand\@maketitle{%
\begin{center}
    \bfseries
    \centering
    \LARGE CSCI 564: \@title
\end{center}
\bigskip
\begin{center}
\makebox[0.8\textwidth]{Name:\enspace\hrulefill\hrulefill\enspace{}CWID:\enspace\hrulefill}
\end{center}
}
\makeatother
\cfoot{Page \thepage\ of \numpages}

\begin{document}
\maketitle
\thispagestyle{foot}

\textbf{For each of the following questions, please show all of your work, and
explain your answers.}

\begin{questions}
    \question Cache performance is a factor of several parameters. For each of
    the following, describe the issues that arise if the value is either too
    small or too large:
    \begin{parts}
        \part[3] Cache size
            \vspace{\stretch{1}}

        \part[3] Line size
            \vspace{\stretch{1}}

        \part[3] Associativity
            \vspace{\stretch{5}}
    \end{parts}

    \newpage
    \question Consider the \texttt{matrix\_add} function shown below:

    \begin{minted}{c}
        int matrix_add(int a[128][128], int b[128][128], int c[128][128]) {
            int i, j;
            for(i = 0; i < 128; i++)
                for(j = 0;j < 128; j++)
                    c[i][j] = a[i][j] + b[i][j];
            return 0;
        }
    \end{minted}

    In each iteration, the compiled code will load \texttt{a[i][j]} first, and
    then load \texttt{b[i][j]}. After performing the sum of those two values,
    the result will be stored in \texttt{c[i][j]}.

    The processor has a 64 KiB, 2-way, 64 byte-block L1 data cache, and the
    cache uses an LRU policy for determining which cache line to evict if a set
    is full. The L1 data cache is write-back and write-allocate.

    For the following questions, assume that the addresses of the \texttt{a},
    \texttt{b}, and \texttt{c} arrays are \texttt{0x10000}, \texttt{0x20000},
    and \texttt{0x30000}, respectively, and that the cache starts out completely
    empty. \textbf{Explain all of your answers.}

    \begin{parts}
        \part[12.5] What is the L1 data cache miss rate for the
        \texttt{matrix\_add} function?
        How many misses are contributed by compulsory miss? How many misses are conflict misses? 
        \vspace{\stretch{10}}

        \part[12.5] If the L1 hit time is 1 cycle, and the L1 miss penalty is 20
        cycles. What is the average memory access time?
        \vspace{\stretch{1}}
    \end{parts}

    \newpage
    \question[16] You are given a cache that has 16 byte blocks, 512 sets, and
    is 2-way set associative. Integers are 4 bytes. Give the C code for a loop
    that has a 100\% miss rate in this cache but whose hit rate rises to almost
    100\% if you double the size of the cache. \textbf{Do not assume the
    starting indexes of any arrays.}

\end{questions}

\end{document}
