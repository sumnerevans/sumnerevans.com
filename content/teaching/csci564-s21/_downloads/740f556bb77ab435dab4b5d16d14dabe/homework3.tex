\documentclass{exam}
\usepackage[letterpaper,margin=1in]{geometry}
\usepackage{minted}
\setminted{autogobble,python3,mathescape,linenos,frame=lines,framesep=2mm,fontsize=\footnotesize}
\setlength{\parindent}{0cm}
\usepackage{mdframed}

\title{Homework 3}
\author{}

\makeatletter
\renewcommand\@maketitle{%
\begin{center}
    \bfseries
    \centering
    \LARGE CSCI 564: \@title
\end{center}
\bigskip
\begin{center}
\makebox[0.8\textwidth]{Name:\enspace\hrulefill\hrulefill\enspace{}CWID:\enspace\hrulefill}
\end{center}
}
\makeatother
\cfoot{Page \thepage\ of \numpages}

\begin{document}
\maketitle
\thispagestyle{foot}

\textbf{For each of the following questions, please show all of your work, and
explain your answers.}

\begin{mdframed}
    \textbf{Key Terms}
    \begin{itemize}
        \item \textit{Aliasing} --- the mapping of two or more branches to the
            same entry in the branch predictor data structure.
        \item \textit{Negative interference} --- the branch prediction rate of
            either or both aliased branches is reduced from what they would have
            been without aliasing.
        \item \textit{Positive interference} --- The branch prediction rate for
            either or both aliased branches is increased from what they would
            have been without aliasing.
    \end{itemize}
\end{mdframed}

\begin{questions}
    \question The size of the branch predictor can affect the amount of aliasing
    (two branches mapping to the same entry in a branch prediction table)
    between branches. While this sort of aliasing usually results in negative
    interference, it can sometimes result in positive interference.

    \begin{parts}
        \part[10] Describe a branch T-NT (taken, not-taken) pattern for two
        branches for which aliasing will result in \textit{negative}
        interference.

        \vspace{\stretch{4}}

        \part[10] Describe a branch T-NT (taken, not-taken) pattern for two
        branches for which aliasing will result in \textit{positive}
        interference.

        \vspace{\stretch{4}}

        \part[5] Why is it that aliasing usually results in \textit{negative}
        interference?

        \vspace{\stretch{1}}
    \end{parts}

    \newpage
    \question Assume we have a machine with a typical MIPS 5-stage pipeline that
    uses branch prediction \textit{without} branch delay slots and has a branch
    misprediction penalty of three cycles. One in every five instructions is a
    branch for a certain program, of which 80\% are predicted correctly by our
    branch predictor. Assume all non-branch instructions have a CPI of 1.

    \begin{parts}
        \part[10] How many cycles would it take to execute $n$ instructions?
        \vspace{\stretch{1}}

        \part[15] Now, imagine we instead have a Pentium 4 which has a 20-stage
        pipeline. Because of this, the branch misprediction penalty is now a
        staggering 19 cycles! What would your branch prediction rate have to be
        such that it has the same performance as the MIPS machine from (a)?
        \vspace{\stretch{2}}
    \end{parts}

\end{questions}

\end{document}
