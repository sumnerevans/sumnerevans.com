\documentclass{exam}
\usepackage[letterpaper,margin=1in]{geometry}
\usepackage{minted}
\setminted{autogobble,python3,mathescape,linenos,frame=lines,framesep=2mm,fontsize=\footnotesize}
\setlength{\parindent}{0cm}
\usepackage{mdframed}
\usepackage{multicol}

\title{Homework 4}
\author{}

\makeatletter
\renewcommand\@maketitle{%
\begin{center}
    \bfseries
    \centering
    \LARGE CSCI 564: \@title
\end{center}
\bigskip
\begin{center}
\makebox[0.8\textwidth]{Name:\enspace\hrulefill\hrulefill\enspace{}CWID:\enspace\hrulefill}
\end{center}
}
\makeatother
\cfoot{Page \thepage\ of \numpages}

\begin{document}
\maketitle
\thispagestyle{foot}

\textbf{For each of the following questions, please show all of your work, and
explain your answers.}

\begin{questions}
    \question
    \textbf{Virtual and Physical Addresses.} Consider these three
    configurations:
    \begin{enumerate}
        \item 32-bit operating system, 4 KiB pages, 1 GiB of RAM
        \item 32-bit operating system, 16 KiB pages, 2 GiB of RAM
        \item 64-bit operating system, 16 KiB pages, 16 GiB of RAM
    \end{enumerate}
    \begin{parts}
        \part[15]
        Fill out the following table showing how many bits are needed for the
        virtual address, physical address, virtual page number, physical page
        number, and offset in each of the above configurations.

        \def\arraystretch{2}%
        \begin{tabular}{|c|c|c|c|c|c|}
            \hline
            & \textbf{Virtual Address}
            & \textbf{Physical Address}
            & \textbf{Virtual Page \#}
            & \textbf{Physical Page \#}
            & \textbf{Offset} \\
            \hline
            \textbf{1} & & & & & \\
            \hline
            \textbf{2} & & & & & \\
            \hline
            \textbf{3} & & & & & \\
            \hline
        \end{tabular}

        \vspace{\stretch{2}}

        \part[5]
        What are some advantages of using a larger page size?

        \vspace{\stretch{1}}

        \part[5]
        What are some disadvantages of using a larger page size?

        \vspace{\stretch{1}}
    \end{parts}

    \newpage

    \question[25]
    \textbf{Using the TLB.} As described in the textbook and explained in class,
    virtual memory uses a page table to track the mapping of virtual addresses
    to physical addresses. To speed up this translation, modern processors
    implement a cache of the most recently used translations called the
    \textit{translation lookaside buffer (TLB)}. This exercise shows how the
    page table and the TLB must be updated as addresses are accessed.

    \setlength{\columnseprule}{0.5pt}
    \begin{multicols}{2}
        \textbf{Initial TLB State} {\color{red} (values are base-10)}

        \begin{tabular}{|c|c|c|c|}
            \hline
            \textbf{Valid} & \textbf{Tag} & \textbf{Physical Page \#} & \textbf{LRU} \\
            \hline
            1 & 11 & 12 & 2 \\
            \hline
            1 & 7 & 4 & 3 \\
            \hline
            1 & 3 & 6 & 4 \\
            \hline
            0 & 4 & 9 & 1 \\
            \hline
        \end{tabular}

        The LRU column works as follows: the \textit{older} an entry is, the
        \textit{lower} its LRU number will be. Each time an entry is used, its
        LRU number is set to 4 (since it is now the most-recently used), and the
        other numbers are adjusted downward accordingly.

        \columnbreak

        \textbf{Initial Page Table State} {\color{red} (values are base-10)}

        \begin{tabular}{|c|c|l|}
            \hline
            \textbf{Index} & \textbf{Valid} & \textbf{Physical Page or On Disk} \\
            \hline
            0 & 1 & 5 \\
            \hline
            1 & 0 & Disk \\
            \hline
            2 & 0 & Disk \\
            \hline
            3 & 1 & 6 \\
            \hline
            4 & 1 & 9 \\
            \hline
            5 & 1 & 11 \\
            \hline
            6 & 0 & Disk \\
            \hline
            7 & 1 & 4 \\
            \hline
            8 & 0 & Disk \\
            \hline
            9 & 0 & Disk \\
            \hline
            10 & 1 & 3 \\
            \hline
            11 & 1 & 12 \\
            \hline
        \end{tabular}
    \end{multicols}

    Assume 4 KiB pages and a four-entry fully-associative TLB with an LRU
    replacement policy. If pages must be brought in from disk, give them the
    next largest unused page number (that is, start at 13).

    \textbf{Given the following list is a stream of virtual addresses seen on
    the system, the initial TLB, and initial page table, show the final state of
    the TLB and page table, and specify whether each memory address is a hit in
    the TLB (H), a hit in the page table (TLB miss, M) or a page fault (PF).}

    \begin{multicols}{2}
        \def\arraystretch{1.5}%
        \begin{tabular}{|l|l|}
            \hline
            \textbf{Address} & \textbf{Result (H, M, PF)} \\
            \hline
            \texttt{0x0FFF} & \\
            \hline
            \texttt{0x7A28} & \\
            \hline
            \texttt{0x3DAD} & \\
            \hline
            \texttt{0x3A98} & \\
            \hline
            \texttt{0x1C19} & \\
            \hline
            \texttt{0x1000} & \\
            \hline
            \texttt{0x22D0} & \\
            \hline
        \end{tabular}

        \textbf{Final TLB State}

        \def\arraystretch{1.5}%
        \begin{tabular}{|l|l|l|l|}
            \hline
            \textbf{Valid} & \textbf{Tag} & \textbf{Physical Page \#} & \textbf{LRU} \\
            \hline
            & & & \\
            \hline
            & & & \\
            \hline
            & & & \\
            \hline
            & & & \\
            \hline
        \end{tabular}

        \columnbreak

        \textbf{Final Page Table State}

        \def\arraystretch{1.5}%
        \begin{tabular}{|l|l|l|l|}
            \hline
            \textbf{Index} & \textbf{Valid} & \textbf{Physical Page or On Disk} \\
            \hline
            0 & & \\
            \hline
            1 & & \\
            \hline
            2 & & \\
            \hline
            3 & & \\
            \hline
            4 & & \\
            \hline
            5 & & \\
            \hline
            6 & & \\
            \hline
            7 & & \\
            \hline
            8 & & \\
            \hline
            9 & & \\
            \hline
            10 & & \\
            \hline
            11 & & \\
            \hline
        \end{tabular}
    \end{multicols}

\end{questions}

\end{document}
